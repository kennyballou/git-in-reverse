\documentclass{beamer}
\usetheme{Berlin}
\usecolortheme[light,accent=blue]{solarized}
\usepackage{fontspec}
\usepackage{xunicode}
\usepackage{xltxtra}
\setmainfont{DejaVuSans}
\usepackage{booktabs}
\usepackage{lmodern}
\usepackage{listings}
\usepackage{color}
\usepackage{tikz}
\usetikzlibrary{trees, shapes.misc, arrows}
\usepackage{pgfkeys}
\usepackage{graphicx}
\graphicspath{{./images/}}
\setbeamertemplate{headline}{}

\lstset{%
    basicstyle=\footnotesize\ttfamily,
    breakatwhitespace=false,
    breaklines=true,
    captionpos=b,
    frame=signle,
    keepspaces=true,
    columns=flexible,
    language=Java,
    numbers=left,
    numbersep=5pt,
    numberstyle=\tiny\color{solarizedBase00},
    showspaces=false,
    showstringspaces=false,
    stepnumber=1,
    showtabs=false,
    stringstyle=\color{solarizedMagenta},
    keywordstyle=\color{solarizedCyan},
    commentstyle=\color{solarizedGreen},
    tabsize=2
}

\title{Learning Git± in Reverse}
\subtitle{A Backwards Introduction to the ``information manager from hell''
[e83c51633]}
\author[Ballou]{Kenny Ballou}
\institute[zData]{%
    \inst{}%
    zData, Inc.
}

\AtBeginSection[]{%
    \begin{frame}
    \tableofcontents[
        currentsection,
        sectionstyle=show/shaded,
        subsectionstyle=show/show/hide]
    \end{frame}
}

\begin{document}
% TikZ Styles
\tikzstyle{every node}=[%
    fill=solarizedBase02,
    draw=solarizedBase01,
    thick,
    rounded corners,
    anchor=north,
    sibling distance=6cm]
\tikzstyle{edge from parent}=[%
    solarizedBase00,
    -o,
    thick,
    draw]

%\tikzstyle{edge from parent path}=[%
%    \tikzparentnode.east |- \tikzchildnode.south]

\begin{frame}[label=titleslide]
\titlepage{}
\end{frame}

\begin{frame}
\tableofcontents[subsectionstyle=hide]
\end{frame}

\begin{frame}
\frametitle{Who am I?}
\begin{itemize}
\item{Hacker}
\item{Developer (read gardener)}
\item{Mathematician}
\item{Student}
\end{itemize}

\end{frame}

\section{Introduction}

\begin{frame}
\begin{figure}
\includegraphics[scale=0.45]{xkcd_git.png}
\caption{XKCD on Git\cite{website:xkcd_git_comic}}
\end{figure}
\end{frame}

\begin{frame}
\frametitle{Git±}
\framesubtitle{What is Git?}
\begin{itemize}
\item<2->{Distributed Versioning Control system (D-VCS)}
\item<3->{``A way to manage code''}
\item<4->{``My preferred VCS tool''}
\item<5->{The ``information manager from hell''}
\item<6->{A distributed DAG}
\item<7->{An object store}
\item<8->{An content addressable filesystem}
\item<9->{A key-value store}
\end{itemize}
\end{frame}

\begin{frame}
\frametitle{Git±}
What does Git store?
\begin{itemize}
\item<2->{Objects}
\item<3->{``Packs''}
\end{itemize}
\end{frame}

\begin{frame}
\frametitle{Git Definitions}
\begin{itemize}
\item{Objects}
\item{Trees}
\item{Commits}
\end{itemize}
\end{frame}

\section{Plumbing}
\subsection{Blobs}
\begin{frame}
\frametitle{Git Objects}
\begin{itemize}
\item{ZLIB compressed blob}
\item{Dumb containers, storing provided content}
\item{Created using the \texttt{git-hash-object} plumbing command}
\end{itemize}
\end{frame}

\begin{frame}
\frametitle{Git Objects}
\begin{figure}
\begin{tikzpicture}
    \node[text=solarizedBlue] at (0,  0) {257cc5642}
        child{node[text=solarizedGreen] {foo}};
    \node[text=solarizedBlue] at (3,  0) {5716ca598}
        child{node[text=solarizedGreen] {bar}};
\end{tikzpicture}
\end{figure}
\end{frame}

\begin{frame}[fragile]
\frametitle{\texttt{git-init}}
\lstinputlisting[language=bash,numbers=none]{code/1/1}
\end{frame}

\begin{frame}[fragile]
\frametitle{Results of \texttt{git-init}}
\lstinputlisting[language=bash,numbers=none]{code/1/2}
\end{frame}

\begin{frame}[fragile]
\frametitle{\texttt{git-hash-object}}
\lstinputlisting[language=bash,numbers=none]{code/1/3}
\end{frame}

\begin{frame}[fragile]
\frametitle{\texttt{git-hash-object}}
\lstinputlisting[language=bash,numbers=none]{code/1/4}
\end{frame}

\begin{frame}[fragile]
\frametitle{\texttt{git-cat-file}}
\lstinputlisting[language=bash,numbers=none]{code/1/5}
\end{frame}

\begin{frame}[fragile]
\frametitle{Raw Access to Git Objects}
\lstinputlisting[language=bash,numbers=none]{code/1/6}
\end{frame}

\begin{frame}
\frametitle{Git Object Limitations}
\begin{itemize}
\item{Remembering 40 character SHA's is hard}
\item{What about file names? Where did that go?}
\item{Big files?}
\end{itemize}
\end{frame}

\subsection{Trees}

\begin{frame}
\frametitle{Git Trees}
\begin{itemize}
\item{ZLIB compressed blobs}
\item{Contain references to files and other trees}
\item{Created using the \texttt{git-update-index} and \texttt{git-write-tree}}
\end{itemize}
\end{frame}

\begin{frame}
\frametitle{Git Trees}
\begin{figure}
\begin{tikzpicture}[sibling distance=3cm]
    \node[text=solarizedMagenta]{./}
        child{node[text=solarizedBlue] {blob}
            child {node[text=solarizedGreen] {foo}}}
        child{node[text=solarizedBlue] {blob}
            child {node[text=solarizedGreen] {bar}}};
\end{tikzpicture}
\end{figure}
\end{frame}

\begin{frame}[fragile]
\frametitle{\texttt{git-update-index} and \texttt{git-write-tree}}
\lstinputlisting[language=bash,numbers=none]{code/2/1}
\end{frame}

\begin{frame}[fragile]
\frametitle{Current Git Objects}
\lstinputlisting[language=bash,numbers=none]{code/2/2}
\end{frame}

\begin{frame}
\frametitle{Current Git Objects}
\begin{figure}
\begin{tikzpicture}[sibling distance=3cm]
    \node[text=solarizedMagenta] at (0, 0) {fcf0be4d7}
        child{node[text=solarizedBlue] {257cc5642}
            child{node[text=solarizedGreen]{foo}}};
    \node[text=solarizedBlue] at (3, -1) {5716ca598} child [missing] {};
\end{tikzpicture}
\end{figure}
\end{frame}

\begin{frame}[fragile]
\frametitle{Adding \texttt{bar.txt}}
\lstinputlisting[language=bash,numbers=none]{code/2/3}
\end{frame}

\begin{frame}[fragile]
\frametitle{Git Objects}
\lstinputlisting[language=bash,numbers=none]{code/2/4}
\end{frame}

\begin{frame}
\frametitle{Git Objects, Updated}
\begin{figure}
\begin{tikzpicture}[sibling distance=3cm]
    \node[text=solarizedMagenta] at (0, 0) {b98c9a9f}
        child{node[text=solarizedBlue] {257cc5642}
            child{node[text=solarizedGreen]{foo}}}
        child{node[text=solarizedBlue] {5716ca598}
            child{node[text=solarizedGreen] {bar}}};
\end{tikzpicture}
\end{figure}
\end{frame}

\begin{frame}[fragile]
\frametitle{Modifying Files}
\lstinputlisting[language=bash,numbers=none]{code/2/5}
\end{frame}

\begin{frame}
\frametitle{Git Objects, Updated}
\begin{figure}
\begin{tikzpicture}[sibling distance=3cm]
    \node[text=solarizedMagenta] at (0, 0) {68b75754}
        child{node[text=solarizedBlue] {a3f555b64}
            child{node[text=solarizedGreen]{foo 2}}}
        child{node[text=solarizedBlue] {5716ca598}
            child{node[text=solarizedGreen] {bar}}};
\end{tikzpicture}
\end{figure}
\end{frame}

\begin{frame}[fragile]
\frametitle{Modifying Files}
\framesubtitle{Current Objects}
\lstinputlisting[language=bash,numbers=none]{code/2/6}
\end{frame}

\begin{frame}
\frametitle{Modifying Files}
\framesubtitle{Current Objects}
\begin{figure}
\begin{tikzpicture}[sibling distance=3cm]
    \node[text=solarizedMagenta] at (0, 0) {68b75754}
        child{node[text=solarizedBlue] {a3f555b64}
            child{node[text=solarizedGreen]{foo 2}}}
        child{node[text=solarizedBlue] {5716ca598}
            child{node[text=solarizedGreen] {bar}}};
    \node[text=solarizedMagenta] at (5, 0) {b98c9a9f}
        child{node[text=solarizedBlue] {257cc5642}
            child{node[text=solarizedGreen]{foo}}}
        child{node[text=solarizedBlue] {5716ca598}
            child{node[text=solarizedGreen] {bar}}};
\end{tikzpicture}
\end{figure}
\end{frame}

\begin{frame}
\frametitle{Modifying Files}
\framesubtitle{Current Objects}
\begin{figure}
\begin{tikzpicture}[sibling distance=3cm]
    \node at (0, 0) {Current Tree}
        child{node[text=solarizedMagenta] {68b75754}
            child{node[text=solarizedBlue] {a3f555b64}
                child{node[text=solarizedGreen]{foo 2}}}
            child{node[text=solarizedBlue] {5716ca598}
                child{node[text=solarizedGreen] {bar}}}};
    \node at (5, 0) {Old Tree}
        child{node[text=solarizedMagenta] {b98c9a9f}
        child{node[text=solarizedBlue] {257cc5642}
            child{node[text=solarizedGreen]{foo}}}
        child{node[text=solarizedBlue] {5716ca598}
            child{node[text=solarizedGreen] {bar}}}};
\end{tikzpicture}
\end{figure}
\end{frame}

\begin{frame}
\frametitle{Current Objects}
\framesubtitle{The Beginnings of a DAG}
\begin{figure}
\begin{tikzpicture}[sibling distance=3cm]
    \node at (0, 0) {Current Tree}
        child{node[text=solarizedMagenta] {68b75754}
            child{node[text=solarizedBlue] {a3f555b64}
                child{node[text=solarizedGreen]{foo 2}}}
            child{node[text=solarizedBlue,dashed] {5716ca598}
                child{node[text=solarizedGreen,dashed] {bar}}}};
    \node at (5, 0) {Old Tree}
        child{node[text=solarizedMagenta] {b98c9a9f}
        child{node[text=solarizedBlue] {5716ca598}
            child{node[text=solarizedGreen] {bar}}}
        child{node[text=solarizedBlue] {257cc5642}
            child{node[text=solarizedGreen]{foo}}}};
\end{tikzpicture}
\end{figure}
\end{frame}

\begin{frame}
\frametitle{Limitation of Git± Trees}
\begin{itemize}
\item{Remembering SHA's is \textit{still} hard}
\item{No metadata about the who, when, and why}
\end{itemize}
\end{frame}

\subsection{Commits}

\begin{frame}
\frametitle{Git± Commit Objects}
\begin{itemize}
\item{ZLIB compressed blob}
\item{Stores metadata about changes}
\item{Stores a reference to the tree being saved}
\end{itemize}
\end{frame}

\begin{frame}[fragile]
\frametitle{\texttt{git-commit-tree}}
\lstinputlisting[language=bash,numbers=none]{code/3/1}
\end{frame}

\begin{frame}
\frametitle{\texttt{git-commit-tree}}
\begin{figure}
\begin{tikzpicture}
    \node[text=solarizedCyan] {a619a045a}
        child{node[text=solarizedMagenta] {fcf0be4d7}
            child{node[text=solarizedBlue] {257cc5642}
                child{node[text=solarizedGreen]{foo}}}};
\end{tikzpicture}
\end{figure}
\end{frame}

\begin{frame}[fragile]
\frametitle{\texttt{git-commit-tree}}
\lstinputlisting[language=bash,numbers=none]{code/3/2}
\end{frame}

\begin{frame}
\frametitle{\texttt{git-commit-tree}}
\begin{figure}
\begin{tikzpicture}[sibling distance=3cm]
    \node[text=solarizedCyan] {2de9adf2b}
        child{node[text=solarizedCyan] {a619a045a}}
        child{node[text=solarizedMagenta] {b98c9a9f9}
            child{node[text=solarizedBlue] {5716ca598}
                child{node[text=solarizedGreen] {bar}}}
            child{node[text=solarizedBlue] {257cc5642}
                child{node[text=solarizedGreen]{foo}}}};
\end{tikzpicture}
\end{figure}
\end{frame}

\begin{frame}[fragile]
\frametitle{\texttt{git-commit-tree}}
\lstinputlisting[language=bash,numbers=none]{code/3/3}
\end{frame}

\begin{frame}
\frametitle{\texttt{git-commit-tree}}
\begin{figure}
\begin{tikzpicture}[sibling distance=3cm]
    \node[text=solarizedCyan] {e0ea04248}
        child{node[text=solarizedCyan] {2de9adf2b}}
        child{node[text=solarizedMagenta] {68b757546}
            child{node[text=solarizedBlue] {5716ca598}
                child{node[text=solarizedGreen] {bar}}}
            child{node[text=solarizedBlue] {a3f555b64}
                child{node[text=solarizedGreen]{foo 2}}}};
\end{tikzpicture}
\end{figure}
\end{frame}

\begin{frame}[fragile]
\frametitle{Git± History}
\lstinputlisting[language=bash,numbers=none]{code/3/4}
\end{frame}

\begin{frame}
\frametitle{Git± Thus Far}
\begin{figure}
\begin{tikzpicture}
    \node[text=solarizedBlue] (foo) at (10, -10) {foo};
\end{tikzpicture}
\end{figure}
\end{frame}

\begin{frame}
\frametitle{Git± Thus Far}
\begin{figure}
\begin{tikzpicture}
    \node[text=solarizedBlue] (foo) at (10, -10) {foo};
    \node[text=solarizedBlue] (hfoo) at (10, -8) {257cc564};
    \draw[-o] (hfoo) -- (foo);
\end{tikzpicture}
\end{figure}
\end{frame}

\begin{frame}
\frametitle{Git± Thus Far}
\begin{figure}
\begin{tikzpicture}
    \node[text=solarizedBlue] (foo) at (10, -10) {foo};
    \node[text=solarizedBlue] (hfoo) at (10, -8) {257cc564};
    \node[text=solarizedMagenta] (tree1) at (9, -6) {fcf0be4d};
    \draw[-o] (hfoo) -- (foo);
    \draw[-o] (tree1) -- (hfoo);
\end{tikzpicture}
\end{figure}
\end{frame}

\begin{frame}
\frametitle{Git± Thus Far}
\begin{figure}
\begin{tikzpicture}
    \node[text=solarizedBlue] (foo) at (10, -10) {foo};
    \node[text=solarizedBlue] (hfoo) at (10, -8) {257cc564};
    \node[text=solarizedMagenta] (tree1) at (9, -6) {fcf0be4d};
    \node[text=solarizedCyan] (commit1) at (5, -6) {a619a045a};
    \draw[-o] (hfoo) -- (foo);
    \draw[-o] (tree1) -- (hfoo);
    \draw[-o] (commit1) -- (tree1);
\end{tikzpicture}
\end{figure}
\end{frame}

\begin{frame}
\frametitle{Git± Thus Far}
\begin{figure}
\begin{tikzpicture}
    \node[dashed,text=solarizedBlue] (foo) at (10, -10) {foo};
    \node[text=solarizedBlue] (bar) at (8,  -10) {bar};
    \node[dashed,text=solarizedBlue] (hfoo) at (10, -8) {257cc564};
    \node[text=solarizedBlue] (hbar) at (8,  -8) {5716ca59};
    \node[text=solarizedMagenta] (tree1) at (9, -6) {fcf0be4d};
    \node[text=solarizedCyan] (commit1) at (5, -6) {a619a045a5};
    \draw[-o] (hfoo) -- (foo);
    \draw[-o] (hbar) -- (bar);
    \draw[-o] (tree1) -- (hfoo);
    \draw[-o] (commit1) -- (tree1);
\end{tikzpicture}
\end{figure}
\end{frame}

\begin{frame}
\frametitle{Git± Thus Far}
\begin{figure}
\begin{tikzpicture}
    \node[dashed,text=solarizedBlue] (foo) at (10, -10) {foo};
    \node[text=solarizedBlue] (bar) at (8,  -10) {bar};
    \node[dashed,text=solarizedBlue] (hfoo) at (10, -8) {257cc564};
    \node[text=solarizedBlue] (hbar) at (8,  -8) {5716ca59};
    \node[text=solarizedMagenta] (tree2) at (9, -6) {b98c9a9f9};
    \node[text=solarizedCyan] (commit2) at (5, -6) {2de9adf2b};
    \node[text=solarizedCyan] (commit1) at (5, -8) {a619a045a5};
    \draw[-o] (hfoo) -- (foo);
    \draw[-o] (hbar) -- (bar);
    \draw[-o] (tree2) -- (hfoo);
    \draw[-o] (tree2) -- (hbar);
    \draw[-o] (commit2) -- (tree2);
    \draw[-o] (commit2) -- (commit1);
\end{tikzpicture}
\end{figure}
\end{frame}

\begin{frame}
\frametitle{Git± Thus Far}
\begin{figure}
\begin{tikzpicture}
    \node[text=solarizedBlue] (foo) at (10, -10) {foo 2};
    \node[dashed,text=solarizedBlue] (bar) at (8,  -10) {bar};
    \node[text=solarizedBlue] (hfoo) at (10, -8) {a3f555b6};
    \node[dashed,text=solarizedBlue] (hbar) at (8,  -8) {5716ca59};
    \node[text=solarizedMagenta] (tree3) at (9, -6) {68b75754};
    \node[text=solarizedCyan] (commit3) at (5, -6) {e0ea04248};
    \node[text=solarizedCyan] (commit2) at (5, -8) {2de9adf2b};
    \node[text=solarizedCyan] (commit1) at (5, -10) {a619a045a5};
    \draw[-o] (hfoo) -- (foo);
    \draw[-o] (hbar) -- (bar);
    \draw[-o] (tree2) -- (hfoo);
    \draw[-o] (tree2) -- (hbar);
    \draw[-o] (commit3) -- (tree3);
    \draw[-o] (commit3) -- (commit2);
    \draw[-o] (commit2) -- (commit1);
\end{tikzpicture}
\end{figure}
\end{frame}

\section{Using Git±}
\begin{frame}
\frametitle{Using Git±}
\framesubtitle{The porcelain over the pipes}
\begin{itemize}
\item{Plumbing commands are difficult, painful, and error prone}
\item{Thankfully, we have friendly ``porcelain'' commands}
\item{The basics can be covered with \texttt{git-add} and \texttt{git-commit}}
\end{itemize}
\end{frame}

\subsection{git-add}
\begin{frame}
\frametitle{\texttt{git-add}}
Combines:
\begin{itemize}
\item<1->{\texttt{git-hash-object}}
\item<2->{\texttt{git-update-index}}
\end{itemize}
\end{frame}

\begin{frame}
\frametitle{\texttt{git-status}}
\framesubtitle{Short aside about \texttt{git-status}}
\begin{itemize}
\item{The go-to command for peering into the current state of a repository}
\item{Provides information about state of all files}
\begin{itemize}
\item{Currently untracked files}
\item{Currently modified files}
\item{Current state of ``staging'' area}
\end{itemize}
\end{itemize}
\end{frame}

\begin{frame}[fragile]
\frametitle{\texttt{git-add}}
\lstinputlisting[language=bash,numbers=none]{code/4/1}
\end{frame}

\subsection{git-commit}
\begin{frame}
\frametitle{\texttt{git-commit}}
\begin{itemize}
\item{Creates ``commit'' out of current staging area}
\begin{itemize}
\item{Requires a short message}
\item{Will implicitly figure out the parent commit}
\item{Forwards the \texttt{HEAD} pointer and the current branch pointer}
\end{itemize}
\end{itemize}
\end{frame}

\begin{frame}[fragile]
\frametitle{\texttt{git-commit}}
\lstinputlisting[language=bash,numbers=none]{code/4/2}
\end{frame}

\section{Advanced Git±}
\subsection{References}
\begin{frame}
\frametitle{Git± Branches}
\begin{itemize}
\item<2->{Named reference to a commit hash}
\item<3->{Defined in text files under \texttt{./.git/refs}}
\end{itemize}
\end{frame}

\begin{frame}[fragile]
\frametitle{Git± Branches}
\lstinputlisting[language=bash,numbers=none]{code/5/1}
\end{frame}

\begin{frame}[fragile]
\frametitle{\texttt{git-branch}}
Branches can be created with a simple invocation of \texttt{git-branch}:
\lstinputlisting[language=bash,numbers=none]{code/5/2}
\end{frame}

\begin{frame}[fragile]
\frametitle{\texttt{git-checkout}}
After the branch is created, we can switch into that branch with
\texttt{git-checkout}:
\lstinputlisting[language=bash,numbers=none]{code/5/3}
\end{frame}

\begin{frame}[fragile]
\frametitle{\texttt{git-checkout -b}}
Or, we can do all in the same command:
\lstinputlisting[language=bash,numbers=none]{code/5/4}
\end{frame}

\begin{frame}[fragile]
\frametitle{Git± Branches}
Currently, all branches are pointing to the same commit:
\lstinputlisting[language=bash,numbers=none]{code/5/5}
\end{frame}

\begin{frame}[fragile]
\frametitle{Git± Branches}
\lstinputlisting[language=bash,numbers=none]{code/5/6}
\end{frame}

\begin{frame}[fragile]
\frametitle{Git± Branches}
\lstinputlisting[language=bash,numbers=none]{code/5/7}
\end{frame}

\subsection{Synchronization}
\begin{frame}[fragile]
\frametitle{Working with remotes}
\begin{itemize}
\item{Clone a repository from, say, Github, will create the remote reference}
\item{Otherwise, can be created with \texttt{git-remote}}
\end{itemize}
Example Usage:
\lstinputlisting[language=bash,numbers=none]{code/5/8}
\end{frame}

\begin{frame}
\frametitle{Working with remotes}
\begin{itemize}
\item{\texttt{git-clone}: Clone ``remote'' repository}
\begin{itemize}
\item{\texttt{SSH://}: Bi-directional}
\item{\texttt{Git://}: Pull only, not authenticated}
\item{\texttt{HTTP (S)://}: Bi-directional, authenticated, but stupid}
\item{\texttt{File://}: Strange}
\end{itemize}
\item{\texttt{git-push}: Push local changes to remote repository}
\item{\texttt{git-remote}: Utility command for working with remotes}
\item{\texttt{git-pull}: Pull remote changes into working copy}
\end{itemize}
\end{frame}

\begin{frame}[fragile]
\frametitle{\texttt{git-pull} considered harmful}
\begin{itemize}
\item{Standard use of \texttt{git-pull} requires clean working directory}
\item{Will force a merge, if drift between remote and local}
\item{I personally prefer creating an alias for \texttt{git-remote}}
\end{itemize}
\lstinputlisting[
    language=bash,
    numbers=none,
    title=\texttt{\~{}/.gitconfig}]{code/5/9}
\end{frame}

\section*{References}
\begin{frame}[allowframebreaks]
\frametitle{References}
\nocite{*}
\renewcommand{\refname}{}
\bibliographystyle{plain}
\bibliography{references}
\end{frame}

\againframe{titleslide}
\end{document}
